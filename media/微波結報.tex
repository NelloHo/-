%% LyX 2.4.1 created this file.  For more info, see https://www.lyx.org/.
%% Do not edit unless you really know what you are doing.
\documentclass{article}
\usepackage{CJKutf8}
\usepackage[T1]{fontenc}
\usepackage[utf8]{inputenc}
\usepackage{amsmath}
\usepackage{amssymb}
\usepackage{graphicx}
\usepackage{wasysym}

\makeatletter

%%%%%%%%%%%%%%%%%%%%%%%%%%%%%% LyX specific LaTeX commands.
%% A simple dot to overcome graphicx limitations
\newcommand{\lyxdot}{.}


\makeatother

\begin{document}
\begin{CJK}{UTF8}{}%

\part*{1. Introduction}

The experiment mainly utilizes microwaves through instruments such
as waveguides to measure wavelength, power, standing wave ratio (SWR),
and the characteristics of dielectric materials.

\section*{1.1 Microwave}

Microwaves are electromagnetic waves with frequencies ranging from
300 MHz to 300 GHz, commonly used in fields such as communication
and radar.

\section*{1.2 Waveguide}

Waveguides are structures used for transmitting microwaves, typically
made of metal. They usually come in rectangular, square, or elliptical
shapes and are characterized by low loss. Microwaves propagate primarily
through modes, where the electromagnetic field distribution depends
on specific boundary conditions. In this experiment, we focus on the
$TE_{10}$ modes of a rectangular waveguide. This means that when
the electric field varies once along the width of the waveguide and
remains constant along the other direction.Through the following calculation,
we can obtain the electromagnetic field distribution for modes in
the waveguide: Consider a ratangular waveguide in figure 1.1: 
\begin{enumerate}
\item \begin{flushleft}
\begin{figure}[h]
\centering
\label{figure=0000201.1:=000020Rectangular=000020waveguide-1}\includegraphics[scale=0.2]{\string"figure1.1 Rectangular wave guide\string".png}

figure 1.1: rectangular waveguide 
\end{figure}
 In the TE mode, we can comsider the eletric field z-component $E_{z}=0$,
we can write electric field and magnetic field as:
\par\end{flushleft}

\end{enumerate}
\begin{center}
$\mathbf{E}(x,y,z)=\mathbf{e}(x,y)e^{-j\beta z}$
\par\end{center}

\begin{center}
$\mathbf{H}(x,y,z)=[\mathbf{h}(x,y)+\hat{z}h_{z}(x,y)]e^{-j\beta z}$
\par\end{center}

\begin{flushleft}
Where $\mathbf{e}(x,y),\mathbf{h}(x,y)$ are the $\hat{x},\hat{y}$
component eletric and magnetic field, and $+\beta$ means propagate
in $+z$ direction.Then we plug them into these two Maxwell's equations:
\par\end{flushleft}

\begin{center}
$\nabla\times\mathbf{E=-}j\omega\mu\mathbf{H}$
\par\end{center}

\begin{center}
$\nabla\times\mathbf{H=}j\omega\epsilon\mathbf{E}$
\par\end{center}

\begin{flushleft}
by some calculation we get this differentuial equation:
\par\end{flushleft}

\begin{center}
$\left(\dfrac{\partial^{2}}{\partial x^{2}}+\dfrac{\partial^{2}}{\partial y^{2}}+\dfrac{\partial^{2}}{\partial z^{2}}+k^{2}\right)H_{z}=0$,
\par\end{center}

\begin{flushleft}
since$H_{z}=h(x,y)e^{-j\beta z}$ ,this can be reduced into two deminsional
wave equation:
\par\end{flushleft}

\begin{center}
$\left(\dfrac{\partial^{2}}{\partial x^{2}}+\dfrac{\partial^{2}}{\partial y^{2}}+k_{c}\right)h_{z}(x,y)=0-(\varhexstar)$
\par\end{center}

\begin{flushleft}
where $k_{c}=\sqrt{k^{2}-\beta^{2}}$, this is called cutoff frequency.
\par\end{flushleft}

\begin{flushleft}
2. We solve the equation $\varhexstar$ by using the method of seperation
of variables, we have the general solution
\par\end{flushleft}

\begin{center}
$h_{z}=(Acosk_{x}x+Bsink_{x}x)(Ccosk_{y}y+Dsink_{y}y)$
\par\end{center}

\begin{flushleft}
and we apply the boundary conditions and Maxwell's equation:
\par\end{flushleft}

\begin{center}
$\begin{cases}
e_{x}(x,y)=0 & (y=0,b)\\
e_{y}(x,y)=0 & (x=0,a)
\end{cases}$ 
\par\end{center}

\begin{center}
$\nabla\times\mathbf{H=}j\omega\epsilon\mathbf{E}$
\par\end{center}

\begin{flushleft}
we can get 
\par\end{flushleft}

\begin{center}
$k_{x}=\dfrac{m\pi}{a},k_{y}=\dfrac{n\pi}{b}(m,n\in\mathbb{N})$ 
\par\end{center}

\begin{flushleft}
In this experiment $m=1,n=0$, so we finally get the answer:
\par\end{flushleft}

\begin{center}
$H_{z}(x,y,z)=A_{mn}cos(\dfrac{\pi x}{a})e^{-j\beta z}$
\par\end{center}

\begin{center}
$E_{x}(x,y,z)=0$
\par\end{center}

\begin{center}
$E_{y}(x,y,z)=\dfrac{-j\omega\mu\pi}{k_{c}^{2}a}A_{mn}sin(\dfrac{\pi x}{a})e^{-j\beta z}$
\par\end{center}

\begin{center}
$H_{x}(x,y,z)=\dfrac{j\beta\pi}{k_{c}^{2}}A_{mn}sin(\dfrac{\pi x}{a})e^{-j\beta z}$
\par\end{center}

\begin{center}
$H_{y}(x,y,z)=0$
\par\end{center}

\begin{center}
$\beta=\sqrt{k^{2}-k_{c}^{2}}=\sqrt{k^{2}-(\dfrac{\pi}{a})^{2}}$
\par\end{center}

\begin{flushleft}
$\beta$ may be real number when $k>k_{c}$,
\par\end{flushleft}

\begin{center}
$f=\dfrac{k_{c}}{2\pi\sqrt{\mu\epsilon}}=\dfrac{1}{2\pi\sqrt{\mu\epsilon}}\sqrt{\dfrac{\pi}{a}}$
\par\end{center}

\begin{flushleft}
wave impedence:
\par\end{flushleft}

\begin{center}
$Z_{TE}=\dfrac{E_{x}}{H_{y}}=\dfrac{-E_{y}}{H_{x}}=\dfrac{k\eta}{\beta}\hphantom{}(\eta=\sqrt{\dfrac{\mu}{\epsilon}})$
\par\end{center}

3. For $TE_{10}$ mode, the eletctromagnetic field distribution like
figure 1.2:

\begin{figure}[h]
\centering
\begin{centering}
\label{Figure1.2=000020TE10=000020mode}
\par\end{centering}
\includegraphics[scale=0.4]{\string"figutre1.2 waveguide mode TE10\string".PNG}
\centering{}figure1.2: $TE_{10}$ mode eletctromagnetic field distribution
\end{figure}


\section*{1.3 Standing Wave Ratio (SWR)}

When an electromagnetic wave passes through two transmission lines
or waveguides with different impedances, reflection and transmission
occur, resulting in the formation of standing waves, which reduce
the transmitted power. Therefore, we can measure the standing wave
ratio (SWR) to quantify the degree of impedance matching. When the
$SWR=1$, it indicates perfect impedance matching and no reflected
waves.
\clearpage
\end{CJK}

\end{document}
